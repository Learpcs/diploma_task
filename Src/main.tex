\documentclass[bachelor, och, assignment, times, article]{../Include/SCWorks}
\usepackage{cmap}
\usepackage[T2A]{fontenc}
\usepackage[utf8]{inputenc}
\usepackage{graphicx}
\usepackage[sort,compress]{cite}
\usepackage{amsmath}
\usepackage[english,russian]{babel}
\usepackage{tempora}
\usepackage[colorlinks=false]{hyperref}
\usepackage{listings}
\usepackage{amssymb}
\usepackage{float} 
\usepackage{amsthm}


\begin{document}
\chair{математического обеспечения вычислительных комплексов и информационных систем}
\title{Автоматическое распознавание фонем на основе современных моделей машинного обучения}
\course{4}
\group{441}
\department{факультета компьютерных наук и информационных технологий}
\napravlenie{02.03.03~--- Математическое обеспечение и администрирование информационных систем \newline профиль «Технологии программирования»}
\studenttitle{студента}
\author{Родина Ивана Сергеевича}
\date{2025}

\chtitle{МОВКиИС} % степень, звание
\chname{Андрейченко\,Д.\,К.}
\satitle{Д-р физ-мат. наук} %должность, степень, звание
\saname{Андрейченко\,Д.\,К.}


\maketitle

%==============
Цель работы: разработать систему распознавания фонем английского языка.

Задачи:
\begin{enumerate}
    \item Исследовать принципы обработки речевых сигналов.
    \item Разобрать методы перевода речевых сигналов в текст.
    \item Провести анализ существующих решений, которые используют методы глубокого обучения.
    \item Разобрать фонетические алфавиты, которые используются для обозначения фонем.
    \item Рассмотреть применение рекуррентных нейронные сетей и трансформеров для перевода речевых сигналов в фонемы
    \item Провести анализ моделей машинного обучения и выбрать лучшего кандидата.
    \item Найти и подготовить данные для моделей машинного обучения.
    \item Придумать эффективную архитектуру для работы веб-сервиса с моделью.
    \item Реализовать прототип веб-сервиса для демонстрации работы модели.
    \item Протестировать систему на реальных аудиозаписях, выявить ограничения и возможные улучшения.
\end{enumerate}

В теоретической части необходимо исследовать задачу перевода аудио в последовательность символов международного фонетического алфавита. Требуется решить данную задачу, используя рекуррентные нейронные сети и трансформеры. Нужно показать актуальность и востребованность данной задачи, рассмотреть существующие решения.

В практической части необходимо придумать архитектуру модели машинного обучения и веб-сервиса, найти и подготовить данные для обучения, и реализовать придуманные архитектуры. 

%=====================
\textbf{Срок предоставления работы:   \underline{\hspace{1.00cm}}.\underline{\hspace{1.25cm}}. 2025 года}.
                                                    
Рассмотрено и одобрено на заседании кафедры математического обеспечения вычислительных комплексов и информационных систем

Протокол № \underline{\hspace{1.00cm}}  от  \underline{\hspace{0.75cm}}.\underline{\hspace{1.00cm}}.2025 года. 

Секретарь  \underline{\hspace{4.00cm}}Заяц Н.А.

Дата выдачи задания  \underline{\hspace{4.00cm}}

Задание получил  \underline{\hspace{4.00cm}} Родин И. С.



\end{document}
